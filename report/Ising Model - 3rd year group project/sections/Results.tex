\section{Results}

\subsection{Autocorrelation}
\label{sec:auto}

In order to compare the speed and effectiveness of each algorithm presented in Section \ref{monte-carlo} we will consider and benchmark the autocorrelation times of the programmes against each other (to determine how fast measurements become decorrelated). We will take the total magnetisation, $M$ as our observable. Let $O_{i} = |M|_{i} - \langle |M| \rangle$, where $|M|_{i}$ is the \textit{i-th} measurement of $|M|$. Hence, autocorrelation function $C(t)$ of $|M|$ is defined as:

\begin{equation}
	\label{eq:autocorrelation}
	C(t) \equiv \frac{\frac{1}{N-t} \sum_{i} O_{i}O_{i+t}}{\langle O^2 \rangle} \propto e^{-t/\tau}
\end{equation}

where one could expect statistical independece as $t \to \infty$ and the autocorrelation function \ref{eq:autocorrelation} apporaches $0$ as $t \to \infty$. 

In Figure \ref{fig:autocorrelation} we will consider the autocorrelation functions of the three algorithms presented in this paper on a $64 \times 64$ lattice. 

\begin{figure}[h!]

  \includegraphics[width=0.6\textwidth]{figures/autocorrelation.png}
  \centering
  \caption{Autocorrelation functions plotted against evolution time for three different Monte-Carlo algorithms: Metropolis, Swendsen-Wang and Wolff measured at the critical temperature, $T_{crit} = 2.269$}
   \label{fig:autocorrelation}
\end{figure}

It appears clear that the autocorrelation time of the Metropolis Algorithm is significantly larger than of the cluster counterparts. single-cluster Wolff algorithm produces the shortest autocorrelation time.

In terms of absolute CPU computation times, cluster algorithms also outcompete the single-spin-flip Metropolis algorithm as shown in Table \ref{tab:methods}:

\begin{table}[h!]
	\centering
	\begin{tabular}{|c |c |c|}
		\hline
		\textbf{Method} & \textbf{time(ms)/iteration}	 &\textbf{$\tau$}\\
		\hline
		Swendsen-Wang 	& 0.14 	& 6.4\\
		\hline
		Wolff		& 0.17	& 1.6\\
		\hline
		Metropolis & 1.58	& 143\\
		\hline	
	\end{tabular}
	\caption{Measured in real (cpu/wall-clock) time, Cluster algorithms are compared to the Metropolis (at the critical temperature $\beta \approx 0.44$)}
	\label{tab:methods}
\end{table}

We will also examine how well they cover the phase space of the spins and finally we will measure the computation times for $64 \times 64$ lattice simulated with each of the algorithms.

\begin{figure}[h!]
  \label{fig:config}
  \includegraphics[width=0.6\textwidth]{figures/configs_compare.png}
  \centering
  \caption{Total absolute magnetisation values of the $64 \times 64$ lattice at each iteration of the algorithms. }
\end{figure}

\subsection{Energy Measurements}

From Section \ref{sec:auto} we have determined that single-cluster Wolff algorithm is best suited to perform Ising Model simulation, therefore we will determine the physical observables through it. Several lattice sizes are used, namely, $8 \times 8$, $16 \times 16$, $32 \times 32$ and $64 \times 64$.



\subsection{Magnetisation Measurements}

\subsection{Error Analysis}

To avoid the uncertainty of recording correlated errors (which makes the whole analysis wrong), one can use \textit{the blocking method}, which 

\subsection{Critical Temperature}

\subsubsection*{Finite size scaling}

\subsubsection*{Binder ratios}
